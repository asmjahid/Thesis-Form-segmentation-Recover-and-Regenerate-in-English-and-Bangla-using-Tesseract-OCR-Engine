\chapter {Performance Analysis}
\label{chap:result}
As we have described the proposed method in the chapter \ref{chap:algorithm}. In this chapter we will discuss about result and performance.
\section{Environments of Experiments}
For this thesis computer generated forms are used to analysis the performance and forms images are captured by a computer generated system. This thesis is started form a scratch level, so computer generated forms are used to analysis the performance. For computer generated forms the results are quite good and the system doing quite well. But it can be assumed that for the high regulation images such as which images are captured by DSLR camera the output will become more perfect. Because DSLR camera has better picture quality. Image of DSLR camera is less noise free. For this we’ll test in different categories images in future.  
   
\section{Performance for English Form}
We test our proposed algorithm in different types of sample form. The result depends on the tesseract training data. For english we train "Arial","Courier","Calibri","Times New Roman" fonts. Here we show our some output result and analysis of our training data.
\subsection{English Sample Form 1}

\begin{figure}[H]
\centering
\includegraphics[width=1\textwidth]{form1.png}
\caption {Sample English form 1}
\label {fig:form1}
\end{figure}

\begin{table}[H]
\centering
\begin{tabular}{|p{2cm}|p{2cm}|p{2cm}|}
\hline
character & Input Frequency & Output Frequency \\
\hline
. & 1 & 0\\
\hline
1 & 5 & 7\\
\hline
0 & 3 & 3\\
\hline
2 & 3 & 3\\
\hline
5 & 2 & 2\\
\hline
8 & 0 & 4\\
\hline
: & 10 & 10\\
\hline
A & 1 & 1\\
\hline
C & 3 & 3\\
\hline
B & 1 & 1\\
\hline
E & 2 & 1\\
\hline
D & 1 & 1\\
\hline
F & 2 & 2\\
\hline
I & 0 & 2\\
\hline
H & 1 & 2\\
\hline
N & 1 & 1\\
\hline
P & 1 & 1\\
\hline
S & 3 & 3\\
\hline
R & 2 & 2\\
\hline
T & 3 & 3\\
\hline
Y & 1 & 2\\
\hline
a & 10 & 9\\
\hline
c & 5 & 5\\
\hline
e & 20 & 18\\
\hline
d & 3 & 3\\
\hline
g & 6 & 0\\
\hline
i & 10 & 8\\
\hline
m & 6 & 6\\
\hline
l & 6 & 1\\
\hline
o & 6 & 6\\
\hline
n & 11 & 7\\
\hline
p & 3 & 3\\
\hline
s & 4 & 5\\
\hline
r & 11 & 11\\
\hline
u & 1 & 1\\
\hline
t & 9 & 10\\
\hline
v & 1 & 1\\
\hline
\end{tabular}
\caption {Frequency table for input \& output of English sample form-1}
\label {tab:Table1}
\end{table}

\begin{figure}[H]
\centering
\includegraphics[width=1\textwidth]{form1.pdf}
\caption {Bar chart input output frequency of English sample form 1}
\label {fig:bar1}
\end{figure}
According to the table \ref{tab:Table1} \& bar chart \ref{fig:bar1} we can say that, input and output frequency of sample English form-01 for the characters 0, 2, 5 ,: ,A ,B ,C ,D ,F ,I ,N ,P ,R ,S ,T ,Y ,a ,c ,d ,l ,o ,p ,r ,s ,t ,u ,v are perfect. But for the 1, 8, E, H, i, characters output frequencies are not perfect. 
\subsection{English Sample Form 2}

\begin{figure}[H]
\centering
\includegraphics[width=1\textwidth]{form2.png}
\caption {Sample English form 2}
\label {fig:form2}
\end{figure}

\begin{table}[H]
\centering
\begin{tabular}{|p{2cm}|p{2cm}|p{2cm}|}
\hline
character & Input Frequency & Output Frequency \\
\hline
. & 1 & 0\\
\hline
1 & 5 & 7\\
\hline
0 & 2 & 1\\
\hline
3 & 4 & 4\\
\hline
2 & 3 & 3\\
\hline
8 & 0 & 4\\
\hline
: & 10 & 10\\
\hline
A & 1 & 1\\
\hline
C & 3 & 3\\
\hline
B & 2 & 2\\
\hline
E & 2 & 1\\
\hline
D & 1 & 1\\
\hline
F & 2 & 2\\
\hline
I & 0 & 4\\
\hline
H & 0 & 1\\
\hline
K & 2 & 2\\
\hline
O & 0 & 1\\
\hline
N & 1 & 1\\
\hline
P & 1 & 1\\
\hline
S & 3 & 3\\
\hline
R & 2 & 2\\
\hline
T & 2 & 2\\
\hline
Y & 1 & 2\\
\hline
a & 9 & 8\\
\hline
c & 5 & 5\\
\hline
e & 20 & 18\\
\hline
d & 3 & 3\\
\hline
g & 6 & 0\\
\hline
i & 9 & 7\\
\hline
m & 7 & 7\\
\hline
l & 6 & 1\\
\hline
o & 6 & 6\\
\hline
n & 10 & 6\\
\hline
p & 4 & 4\\
\hline
s & 3 & 4\\
\hline
r & 12 & 10\\
\hline
u & 3 & 3\\
\hline
t & 9 & 10\\
\hline
\end{tabular}
\caption {Frequency table for input \& output of English sample form-2}
\label {tab:Table2}
\end{table}

\begin{figure}[H]
\centering
\includegraphics[width=1\textwidth]{form2.pdf}
\caption {Bar chart input output frequency of sample form-2}
\label {fig:bar2}
\end{figure}
According to the table \ref{tab:Table2} \& bar chart \ref{fig:bar2} we can say that, For sample English form-02 the input and output frequencies are perfect for the characters 2, 3 , :  , A , B, C, D, F, K, N, P, R, T, e, m, u. For the characters 0, 1, 8, E, H, I, O, S, Y, a, c, i, n, o , p, r, s, t outputs are quite different. 
\subsection{English Sample Form 3}

\begin{figure}[H]
\centering
\includegraphics[width=1\textwidth]{form3.png}
\caption {Sample English form 3}
\label {fig:form3}
\end{figure}

\begin{table}[H]
\centering
\begin{tabular}{|p{2cm}|p{2cm}|p{2cm}|}
\hline
character & Input Frequency & Output Frequency \\
\hline
. & 1 & 0\\
\hline
1 & 5 & 7\\
\hline
0 & 2 & 1\\
\hline
2 & 3 & 3\\
\hline
4 & 2 & 2\\
\hline
6 & 2 & 2\\
\hline
8 & 0 & 4\\
\hline
: & 10 & 10\\
\hline
C & 4 & 4\\
\hline
B & 1 & 1\\
\hline
E & 2 & 1\\
\hline
D & 1 & 1\\
\hline
F & 1 & 1\\
\hline
I & 0 & 2\\
\hline
H & 1 & 2\\
\hline
O & 0 & 1\\
\hline
N & 2 & 2\\
\hline
P & 1 & 1\\
\hline
S & 5 & 5\\
\hline
R & 2 & 2\\
\hline
U & 1 & 1\\
\hline
T & 2 & 2\\
\hline
V & 0 & 2\\
\hline
Y & 1 & 2\\
\hline
a & 9 & 8\\
\hline
c & 6 & 6\\
\hline
e & 23 & 21\\
\hline
d & 4 & 4\\
\hline
g & 6 & 0\\
\hline
i & 9 & 7\\
\hline
m & 7 & 7\\
\hline
l & 6 & 1\\
\hline
o & 8 & 8\\
\hline
n & 11 & 7\\
\hline
p & 3 & 3\\
\hline
s & 4 & 5\\
\hline
r & 9 & 9\\
\hline
u & 1 & 1\\
\hline
t & 8 & 9\\
\hline
y & 1 & 1\\
\hline
\end{tabular}
\caption {Frequency table for input \& output of English sample form-3}
\label {tab:Table3}
\end{table}

\begin{figure}[H]
\centering
\includegraphics[width=1\textwidth]{form3.pdf}
\caption {Bar chart input output frequency of sample form 3}
\label {fig:bar3}
\end{figure}
According to the table \ref{tab:Table3} \& bar chart \ref{fig:bar3} we can say that, for sample English form-3 the input and output frequencies are perfect for the characters 2, 4, 6, :, B, C, D, F, N, P, R, S, T, U, c, d, e, m, n, o, p, r, s, t, u, y for sample English form-03. For the characters 0, 1, 8, E, H, I, O, V, Y, a, i and l input and output frequencies are not equal. 

\subsection{English Sample Form 4}

\begin{figure}[H]
\centering
\includegraphics[width=1\textwidth]{form4.png}
\caption {Sample English form 4}
\label {fig:form4}
\end{figure}

\begin{table}[H]
\centering
\begin{tabular}{|p{2cm}|p{2cm}|p{2cm}|}
\hline
character & Input Frequency & Output Frequency \\
\hline
. & 1 & 0\\
\hline
1 & 5 & 10\\
\hline
0 & 2 & 0\\
\hline
3 & 2 & 2\\
\hline
2 & 5 & 5\\
\hline
8 & 0 & 4\\
\hline
: & 10 & 10\\
\hline
A & 1 & 1\\
\hline
C & 3 & 3\\
\hline
B & 2 & 2\\
\hline
E & 2 & 1\\
\hline
D & 1 & 1\\
\hline
F & 1 & 1\\
\hline
I & 1 & 2\\
\hline
H & 0 & 1\\
\hline
O & 0 & 2\\
\hline
N & 1 & 1\\
\hline
P & 1 & 1\\
\hline
S & 5 & 5\\
\hline
R & 3 & 3\\
\hline
T & 2 & 2\\
\hline
Y & 1 & 2\\
\hline
a & 8 & 8\\
\hline
c & 6 & 6\\
\hline
b & 0 & 1\\
\hline
e & 21 & 19\\
\hline
d & 4 & 4\\
\hline
g & 6 & 0\\
\hline
f & 1 & 1\\
\hline
i & 8 & 6\\
\hline
h & 1 & 0\\
\hline
m & 7 & 7\\
\hline
l & 7 & 1\\
\hline
o & 8 & 8\\
\hline
n & 9 & 6\\
\hline
p & 3 & 3\\
\hline
s & 3 & 3\\
\hline
r & 10 & 8\\
\hline
u & 3 & 3\\
\hline
t & 9 & 10\\
\hline
\end{tabular}
\caption {Frequency table for input \& output of English sample form-4}
\label {tab:Table4}
\end{table}

\begin{figure}[H]
\centering
\includegraphics[width=1\textwidth]{form4.pdf}
\caption {Bar chart input output frequency of sample form 4}
\label {fig:bar4}
\end{figure}

According to the table \ref{tab:Table4} \& bar chart \ref{fig:bar4} we can say that, for sample English form-4 the input and output frequencies are perfect for the characters 2, 3, :, A, C, D, E, F, H, I, N, O, P, R, S, T, Y, a, b, d, f, i, l, m, n, r, s, u for sample English form-03. For the characters ., 0, 1, 8, E, H, I, O, V, Y, a, b, h, n, i, r, t and l input and output frequencies are not equal. 

\subsection{English Sample Form 5}
\begin{figure}[H]
\centering
\includegraphics[width=1\textwidth]{form5.png}
\caption {Sample English form 5}
\label {fig:form4}
\end{figure}

\begin{table}[H]
\centering
\begin{tabular}{|p{2cm}|p{2cm}|p{2cm}|}
\hline
character & Input Frequency & Output Frequency \\
\hline
. & 1 & 0\\
\hline
1 & 5 & 7\\
\hline
0 & 2 & 0\\
\hline
2 & 5 & 5\\
\hline
9 & 2 & 2\\
\hline
8 & 0 & 4\\
\hline
: & 10 & 10\\
\hline
A & 1 & 1\\
\hline
C & 3 & 3\\
\hline
B & 1 & 1\\
\hline
E & 3 & 2\\
\hline
D & 1 & 1\\
\hline
G & 1 & 1\\
\hline
F & 1 & 1\\
\hline
I & 0 & 2\\
\hline
H & 0 & 1\\
\hline
L & 0 & 1\\
\hline
O & 0 & 2\\
\hline
N & 1 & 1\\
\hline
P & 1 & 1\\
\hline
S & 4 & 4\\
\hline
R & 2 & 2\\
\hline
U & 1 & 1\\
\hline
T & 4 & 4\\
\hline
Y & 1 & 2\\
\hline
a & 7 & 7\\
\hline
c & 6 & 6\\
\hline
b & 0 & 1\\
\hline
e & 22 & 20\\
\hline
d & 5 & 5\\
\hline
g & 6 & 0\\
\hline
i & 8 & 5\\
\hline
h & 2 & 2\\
\hline
m & 8 & 9\\
\hline
l & 5 & 1\\
\hline
o & 8 & 9\\
\hline
n & 11 & 8\\
\hline
p & 3 & 3\\
\hline
s & 4 & 3\\
\hline
r & 10 & 8\\
\hline
u & 2 & 2\\
\hline
t & 9 & 8\\
\hline
y & 1 & 1\\
\hline
\end{tabular}
\caption {Frequency table for input \& output of English sample form-5}
\label {tab:Table5}
\end{table}

\begin{figure}[H]
\centering
\includegraphics[width=1\textwidth]{form5.pdf}
\caption {Bar chart input output frequency of sample form 5}
\label {fig:bar5}
\end{figure}
According to the table \ref{tab:Table5} \& bar chart \ref{fig:bar5} we can say that, for sample English form-5 the input and output frequencies are perfect for the characters 2, 9, :, A, C, D, E, F, I, N, P, R, S, T, U, Y, a, c, d, f, i, l, m, n, r, s, u for sample English form-03. For the characters ., 0, 1, 8, E, H, I, O, V, Y, a, b, g, h, n, i, r, o, t and l input and output frequencies are not equal. 
\section{Accuracy for English}
For English, with different types of form \& different types of data we test our system. For some character output accuracy was 100\% correct. For some character output occur more than input character and for some character output occur less than input character. Here we show some accuracy analysis.

\begin{figure}[H]
\centering
\includegraphics[width=1\textwidth]{EOccurance.pdf}
\caption {English miss match character occurrence frequency}
\label {fig:Accuracy}
\end{figure}

In the chart \ref{fig:Accuracy} we show which character gives wrong output. In this bar chart positive axis denoted the character more occurrence than input character and negative axis denoted character less occurrence than input character.

\begin{figure}[H]
\centering
\includegraphics[width=1\textwidth]{EError.pdf}
\caption {English miss match character error rate chart}
\label {fig:Eerror}
\end{figure}

In the chart \ref{fig:Eerror} we show the miss match character error rate. Here we see that character ., 8, L, O, V, b, g gives maximum error.

Considering these kind of error rate, we measure our system accuracy is 75.03\% for english.
\newpage

\section{Performance for Bangla Form}
The result of Bangla form depends on the tesseract training data same as English form. For Bangla we train "Siyam rupali" font. Here we show our some output result and analysis of our training data.
\subsection{Bangla Sample Form 1}
\begin{figure}[H]
\centering
\includegraphics[width=1\textwidth]{formBen01.JPG}
\caption {Sample Bangla form 1}
\label {fig:FormBan1}
\end{figure}

\begin{table}[H]
\centering
\begin{tabular}{|p{2cm}|p{2cm}|p{2cm}|}
\hline
character & Input Frequency & Output Frequency \\
\hline
{\bengalifont ঃ} & 3 & 5\\
\hline
{\bengalifont অ} & 1 & 1\\
\hline
{\bengalifont ই} & 1 & 2\\
\hline
{\bengalifont উ} & 1 & 1\\
\hline
{\bengalifont এ} & 2 & 2\\
\hline
{\bengalifont ক} & 2 & 2\\
\hline
{\bengalifont গ} & 1 & 1\\
\hline
{\bengalifont চ} & 1 & 0\\
\hline
{\bengalifont ত} & 1 & 1\\
\hline
{\bengalifont '} & 8 & 0\\
\hline
{\bengalifont র} & 2 & 3\\
\hline
{\bengalifont ন} & 4 & 4\\
\hline
{\bengalifont প} & 1 & 1\\
\hline
{\bengalifont ভ} & 2 & 2\\
\hline
{\bengalifont ব} & 6 & 5\\
\hline
{\bengalifont য} & 2 & 1\\
\hline
{\bengalifont ম} & 1 & 1\\
\hline
{\bengalifont 0} & 4 & 0\\
\hline
{\bengalifont ল} & 4 & 2\\
\hline
{\bengalifont ষ} & 2 & 2\\
\hline
{\bengalifont শ} & 2 & 2\\
\hline
{\bengalifont হ} & 1 & 1\\
\hline
{\bengalifont স} & 1 & 3\\
\hline
{\bengalifont ;} & 1 & 0\\
\hline
{\bengalifont ়} & 1 & 0\\
\hline
{\bengalifont ি} & 8 & 7\\
\hline
{\bengalifont া} & 7 & 7\\
\hline
{\bengalifont ী} & 0 & 1\\
\hline
{\bengalifont ো} & 1 & 1\\
\hline
{\bengalifont ্} & 6 & 4\\
\hline
{\bengalifont য়} & 0 & 1\\
\hline
{\bengalifont ১} & 4 & 4\\
\hline
{\bengalifont ০} & 0 & 4\\
\hline
{\bengalifont ২} & 2 & 2\\
\hline
{\bengalifont ৫} & 1 & 1\\
\hline
{\bengalifont দ} & 1 & 1\\
\hline
{\bengalifont ৬} & 1 & 1\\
\hline
\end{tabular}
\caption { Frequency table for input \& output of Bangla sample form-1}
\label {tab:BTable1}
\end{table}

\begin{figure}[H]
\centering
\includegraphics[width=1\textwidth]{Bform1.pdf}
\caption {Bar chart input output frequency of Bangla Sample form-1}
\label {fig:Bbar1}
\end{figure}


According to the table \ref{tab:BTable1} \& bar chart \ref{fig:Bbar1} we can say that, input and output frequency of sample Bangla form-01 for the characters {\bengalifont অ}, {\bengalifont উ}, {\bengalifont এ}, {\bengalifont ক}, {\bengalifont গ}, {\bengalifont ত}, {\bengalifont ন},{\bengalifont প},{\bengalifont ভ},{\bengalifont ম},{\bengalifont ষ},{\bengalifont শ},{\bengalifont হ},{\bengalifont া},{\bengalifont ো},{\bengalifont ১},{\bengalifont ২},{\bengalifont ৫},{\bengalifont দ},{\bengalifont ৬} are perfect. But other characters output frequencies are not perfect.

\subsection{Bangla Sample Form 2}
\begin{figure}[H]
\centering
\includegraphics[width=1\textwidth]{formBen02.JPG}
\caption {Sample Bangla form 2}
\label {fig:FormBan2}
\end{figure}

\begin{table}[H]
\centering
\begin{tabular}{|p{2cm}|p{2cm}|p{2cm}|}
\hline
character & Input Frequency & Output Frequency \\
\hline
{\bengalifont ঃ} & 4 & 5\\
\hline
{\bengalifont ই} & 4 & 4\\
\hline
{\bengalifont উ} & 0 & 1\\
\hline
{\bengalifont ঊ} & 1 & 0\\
\hline
{\bengalifont এ} & 2 & 2\\
\hline
{\bengalifont ক} & 1 & 1\\
\hline
{\bengalifont গ} & 1 & 1\\
\hline
{\bengalifont ’} & 1 & 0\\
\hline
{\bengalifont চ} & 1 & 0\\
\hline
{\bengalifont '} & 7 & 0\\
\hline
{\bengalifont দ} & 1 & 1\\
\hline
{\bengalifont ন} & 1 & 1\\
\hline
{\bengalifont ফ} & 1 & 1\\
\hline
{\bengalifont প} & 1 & 1\\
\hline
{\bengalifont ভ} & 1 & 1\\
\hline
{\bengalifont ব} & 6 & 5\\
\hline
{\bengalifont য} & 2 & 1\\
\hline
{\bengalifont ম} & 2 & 2\\
\hline
{\bengalifont র} & 1 & 2\\
\hline
{\bengalifont ল} & 5 & 4\\
\hline
{\bengalifont ষ} & 2 & 2\\
\hline
{\bengalifont শ} & 2 & 2\\
\hline
{\bengalifont স} & 4 & 4\\
\hline
{\bengalifont ়} & 1 & 0\\
\hline
{\bengalifont ি} & 6 & 6\\
\hline
{\bengalifont া} & 6 & 6\\
\hline
{\bengalifont ু} & 1 & 1\\
\hline
{\bengalifont ে} & 1 & 0\\
\hline
{\bengalifont ো} & 1 & 1\\
\hline
{\bengalifont ্} & 6 & 4\\
\hline
{\bengalifont য়} & 0 & 1\\
\hline
{\bengalifont ১} & 4 & 4\\
\hline
{\bengalifont ০} & 3 & 3\\
\hline
{\bengalifont ৩} & 1 & 1\\
\hline
{\bengalifont ২} & 3 & 3\\
\hline
{\bengalifont ৬} & 1 & 1\\
\hline
\end{tabular}
\caption {Frequency table for input \& output of Bangla sample form-2}
\label {tab:BTable2}
\end{table}


\begin{figure}[H]
\centering
\includegraphics[width=1\textwidth]{Bform2.pdf}
\caption {Bar chart input output frequency of Bangla Sample form 2}
\label {fig:Bbar2}
\end{figure}

According to the table \ref{tab:BTable2} \& bar chart \ref{fig:Bbar2} we can say that, input and output frequency of sample Bangla form-01 for the characters {\bengalifont ই, এ, ক, গ, দ, ন, ফ, প, ভ, ম, ষ, শ, স, ি, া, ু, ো, ১, ০, ২,৩,৬ }are perfect. But other characters output frequencies are not perfect.

\subsection{Bangla Sample Form 3}
\begin{figure}[H]
\centering
\includegraphics[width=1\textwidth]{formBen03.JPG}
\caption {Sample Bangla form 3}
\label {fig:FormBan3}
\end{figure}


\begin{table}[H]
\centering
\begin{tabular}{|p{2cm}|p{2cm}|p{2cm}|}
\hline
character & Input Frequency & Output Frequency \\
\hline
{\bengalifont ঃ} & 4 & 5\\
\hline
{\bengalifont ই} & 1 & 2\\
\hline
{\bengalifont আ} & 1 & 1\\
\hline
{\bengalifont উ} & 1 & 2\\
\hline
{\bengalifont ঊ} & 1 & 0\\
\hline
{\bengalifont এ} & 2 & 2\\
\hline
{\bengalifont ক} & 1 & 1\\
\hline
{\bengalifont গ} & 1 & 1\\
\hline
{\bengalifont ’} & 1 & 0\\
\hline
{\bengalifont চ} & 1 & 0\\
\hline
{\bengalifont '} & 8 & 0\\
\hline
{\bengalifont দ} & 1 & 1\\
\hline
{\bengalifont ন} & 1 & 1\\
\hline
{\bengalifont প} & 1 & 1\\
\hline
{\bengalifont ভ} & 1 & 1\\
\hline
{\bengalifont ব} & 9 & 8\\
\hline
{\bengalifont য} & 2 & 1\\
\hline
{\bengalifont ম} & 3 & 3\\
\hline
{\bengalifont 0} & 1 & 0\\
\hline
{\bengalifont ল} & 6 & 4\\
\hline
{\bengalifont ষ} & 2 & 2\\
\hline
{\bengalifont শ} & 2 & 2\\
\hline
{\bengalifont হ} & 2 & 2\\
\hline
{\bengalifont 8} & 1 & 0\\
\hline
{\bengalifont ়} & 1 & 0\\
\hline
{\bengalifont ি} & 7 & 7\\
\hline
{\bengalifont া} & 6 & 6\\
\hline
{\bengalifont ু} & 1 & 1\\
\hline
{\bengalifont ো} & 1 & 1\\
\hline
{\bengalifont ্} & 6 & 4\\
\hline
{\bengalifont র} & 1 & 2\\
\hline
{\bengalifont স} & 0 & 2\\
\hline
{\bengalifont য়} & 0 & 1\\
\hline
{\bengalifont ১} & 4 & 4\\
\hline
{\bengalifont ০} & 2 & 3\\
\hline
{\bengalifont ৩} & 1 & 1\\
\hline
{\bengalifont ২} & 2 & 2\\
\hline
{\bengalifont ৪} & 0 & 1\\
\hline
{\bengalifont ৬} & 1 & 1\\
\hline
\end{tabular}
\caption {Frequency table for input \& output of Bangla sample form-3}
\label {tab:BTable3}
\end{table}

\begin{figure}[H]
\centering
\includegraphics[width=1\textwidth]{Bform3.pdf}
\caption {Bar chart input output frequency of Bangla Sample form-3}
\label {fig:Bbar3}
\end{figure}

According to the table \ref{tab:BTable3} \& bar chart \ref{fig:Bbar3} we can say that, input and output frequency of sample Bangla form-01 for the characters {\bengalifont আ,  এ, ক, গ, দ, ন, প, ভ, ম, ষ, শ, হ, ি, া, ু, ো, ১, ৩, ২, ৬  }are perfect. But other characters output frequencies are not perfect.

\subsection{Bangla Sample Form 4}
\begin{figure}[H]
\centering
\includegraphics[width=1\textwidth]{formBen04.JPG}
\caption {Sample Bangla form 4}
\label {fig:FormBan4}
\end{figure}

\begin{table}[H]
\centering
\begin{tabular}{|p{2cm}|p{2cm}|p{2cm}|}
\hline
character & Input Frequency & Output Frequency \\
\hline
{\bengalifont ঃ} & 4 & 5\\
\hline
{\bengalifont ই} & 1 & 2\\
\hline
{\bengalifont আ} & 1 & 1\\
\hline
{\bengalifont উ} & 0 & 1\\
\hline
{\bengalifont ঊ} & 1 & 0\\
\hline
{\bengalifont এ} & 2 & 2\\
\hline
{\bengalifont ক} & 1 & 1\\
\hline
{\bengalifont গ} & 1 & 1\\
\hline
{\bengalifont ’} & 1 & 0\\
\hline
{\bengalifont চ} & 1 & 0\\
\hline
{\bengalifont '} & 8 & 0\\
\hline
{\bengalifont দ} & 1 & 1\\
\hline
{\bengalifont ন} & 3 & 3\\
\hline
{\bengalifont প} & 1 & 1\\
\hline
{\bengalifont ভ} & 1 & 1\\
\hline
{\bengalifont ব} & 6 & 5\\
\hline
{\bengalifont য} & 2 & 1\\
\hline
{\bengalifont ম} & 2 & 2\\
\hline
{\bengalifont র} & 1 & 2\\
\hline
{\bengalifont ল} & 5 & 3\\
\hline
{\bengalifont ষ} & 2 & 2\\
\hline
{\bengalifont শ} & 2 & 2\\
\hline
{\bengalifont স} & 0 & 2\\
\hline
{\bengalifont ়} & 1 & 0\\
\hline
{\bengalifont ি} & 6 & 6\\
\hline
{\bengalifont া} & 6 & 5\\
\hline
{\bengalifont ে} & 1 & 0\\
\hline
{\bengalifont ো} & 1 & 2\\
\hline
{\bengalifont ্} & 6 & 4\\
\hline
{\bengalifont য়} & 0 & 1\\
\hline
{\bengalifont ১} & 4 & 4\\
\hline
{\bengalifont ০} & 3 & 3\\
\hline
{\bengalifont ৩} & 1 & 1\\
\hline
{\bengalifont ২} & 2 & 2\\
\hline
{\bengalifont ৬} & 2 & 2\\
\hline
\end{tabular}
\caption {Frequency table for input \& output of Bangla sample form-4}
\label {tab:BTable4}
\end{table}

\begin{figure}[H]
\centering
\includegraphics[width=1\textwidth]{Bform4.pdf}
\caption {Bar chart input output frequency of Bangla Sample form-4}
\label {fig:Bbar4}
\end{figure}

According to the table \ref{tab:BTable4} \& bar chart \ref{fig:Bbar4} we can say that, input and output frequency of sample Bangla form-01 for the characters {\bengalifont আ, এ, ক, গ, দ, ন, প, ভ, ম, ষ, শ, ি, ১, ০, ৩, ২, ৬ }are perfect. But other characters output frequencies are not perfect.

\subsection{Bangla Sample Form 5}
\begin{figure}[H]
\centering
\includegraphics[width=1\textwidth]{formBen05.JPG}
\caption {Sample Bangla form 5}
\label {fig:FormBan5}
\end{figure}

\begin{table}[H]
\centering
\begin{tabular}{|p{2cm}|p{2cm}|p{2cm}|}
\hline
character & Input Frequency & Output Frequency \\
\hline
{\bengalifont ঃ} & 5 & 5\\
\hline
{\bengalifont 0} & 1 & 0\\
\hline
{\bengalifont ই} & 1 & 2\\
\hline
{\bengalifont আ} & 1 & 1\\
\hline
{\bengalifont উ} & 1 & 1\\
\hline
{\bengalifont এ} & 3 & 3\\
\hline
{\bengalifont ক} & 1 & 1\\
\hline
{\bengalifont গ} & 1 & 1\\
\hline
{\bengalifont ’} & 1 & 0\\
\hline
{\bengalifont চ} & 1 & 0\\
\hline
{\bengalifont ত} & 1 & 1\\
\hline
{\bengalifont '} & 7 & 0\\
\hline
{\bengalifont দ} & 2 & 2\\
\hline
{\bengalifont ন} & 4 & 3\\
\hline
{\bengalifont প} & 1 & 1\\
\hline
{\bengalifont ভ} & 1 & 1\\
\hline
{\bengalifont ব} & 6 & 5\\
\hline
{\bengalifont য} & 3 & 1\\
\hline
{\bengalifont ম} & 4 & 3\\
\hline
{\bengalifont র} & 1 & 2\\
\hline
{\bengalifont ল} & 4 & 2\\
\hline
{\bengalifont ষ} & 2 & 2\\
\hline
{\bengalifont শ} & 2 & 2\\
\hline
{\bengalifont হ} & 2 & 2\\
\hline
{\bengalifont স} & 1 & 3\\
\hline
{\bengalifont ়} & 2 & 0\\
\hline
{\bengalifont ি} & 6 & 6\\
\hline
{\bengalifont া} & 5 & 4\\
\hline
{\bengalifont ু} & 1 & 1\\
\hline
{\bengalifont ে} & 1 & 1\\
\hline
{\bengalifont ো} & 1 & 1\\
\hline
{\bengalifont ্} & 7 & 5\\
\hline
{\bengalifont য়} & 0 & 2\\
\hline
{\bengalifont ১} & 4 & 4\\
\hline
{\bengalifont ০} & 2 & 3\\
\hline
{\bengalifont ২} & 3 & 3\\
\hline
{\bengalifont ৬} & 1 & 1\\
\hline
{\bengalifont ৯} & 1 & 1\\
\hline
\end{tabular}
\caption {Frequency table for input \& output of Bangla sample form-5}
\label {tab:BTable5}
\end{table}

\begin{figure}[H]
\centering
\includegraphics[width=1\textwidth]{Bform5.pdf}
\caption {Bar chart input output frequency of Bangla Sample form 5}
\label {fig:Bbar5}
\end{figure}

According to the table \ref{tab:BTable5} \& bar chart \ref{fig:Bbar5} we can say that, input and output frequency of sample Bangla form-01 for the characters {\bengalifont আ, এ, ক, গ, দ, প, ভ, ষ, শ, হ, ি, ু, ো, ১, ০, ২, ৬, ৯ }are perfect. But other characters output frequencies are not perfect.
\section{Performance for Bangla Form}
For Bangla, with different types of form \& different types of data we test our system. In Bangla for single character output accuracy was 100\% correct. For "Jukto-Borno" character output occur more than input character and for some "Jukto-Borno" character output occur less than input character. Here we show some accuracy analysis for Bangla.
\begin{figure}[H]
\centering
\includegraphics[width=1\textwidth]{BOccurance.pdf}
\caption {Bangla miss match character occurrence frequency}
\label {fig:BAccuracy}
\end{figure}

In the chart \ref{fig:BAccuracy} we show which character gives wrong output. In this bar chart positive axis denoted the character more occurrence than input character and negative axis denoted character less occurrence than input character. Most of the case "Jukto-Borno" character give wrong output.

\begin{figure}[H]
\centering
\includegraphics[width=1\textwidth]{BError.pdf}
\caption {Bangla miss match character error rate chart}
\label {fig:BEerror}
\end{figure}

In the chart \ref{fig:BEerror} we show the miss match character error rate. Here we see that Bangla character {\bengalifont ০ ঊ ’ চ ' ; ় ী ৪ য়} gives maximum error.

Considering these kind of error rate, we measure our system accuracy is 66.13\% for english.