\chapter{Introduction}
\label{intro}
Segmentation of images \cite{Segmentation} is the widely investigated field of image processing, image analysis and important module of early vision problem. It is the process to cluster a form image into some isolated image regions corresponding to individual surface, objects or some natural part of the object. It is the process of separating an image into some disjoint or distinct regions whose characteristic such as intensity; color texture etc. are similar. No two such regions are similar with respect to these characteristics. in digital image processing, digital image analysis usually involves a "low-level" and a "high-level" processing. In low-level analysis, the representation of an image is transformed from a numerical array of pixel intensities to a symbolic set of image primitives: edges and regions. In high-level analysis, object labels (or interpretations) are assigned to these primitives, thereby providing a semantic description of the image.

An image is segmented \cite{Segmentation} for different kind of implementations like object recognition to extract data from it, occlusion boundary estimation within motion or stereo systems, image compression, image editing or image database lookup.

The output of image segmentation is a set of segments that collectively cover the entire image, or a set of contours extracted from the image. Each of the pixels in a region are similar with respect to some characteristic or computed an image is segmented for different kind of implementations like object recognition to extract data from it, occlusion boundary estimation within motion or stereo systems, image compression, image editing or image database lookup.Adjacent regions are significantly different with respect to the same characteristic.
 
Image segmentation is a fundamental part of the "low level" aspects of computer visions and has many practical applications such as in medical imaging, industrial automation and satellite imagery.Traditional methods for image segmentation have approached the problem either from localization in class space using region information or from localization in position, using edge or boundary information. for monochrome images generally are based on one of two basic properties of gray- level values: discontinuity and similarity. In the first category, the approach is to partition an image based on abrupt changes in gray level. The principal areas of interest within this category are detection of isolated points and detection of lines and edges in an image. The principal approaches in the first category are based on edge detection, and boundary detection. Basically, the idea underlying most edge-detection techniques is the computation of a local derivative operator. The first derivative of the gray-level profile is positive at the leading edge of a transition, negative at the trailing edge, and zero in areas of constant gray level. Hence the magnitude of the first derivative can be used to detect the presence of an edge in an image.
In this paper we are proposing a form segmentation, an image segmentation methodology using threshold techniques that will be applied to a form image and data matching algorithms to detect different objects and clusters.
The goal of segmenting a form image is to extract text data isolating texts of different part of that image by segmenting form into several sections. It becomes easy to extract digital data by processing individual objects into texts if we can first isolate that object scenario and use it for processing.

\section{Objective}
The objective of this thesis is to decompose the image into parts for further analysis. In simple cases, the environment might be well enough controlled so that the segmentation process reliably extracts only the parts that need to be analyzed further. The segmentation is reliable provided that the person's clothing or room background does not have the same color components as a human face. In complex cases, such as extracting a complete road network from a grayscale aerial image, the segmentation problem can be very difficult and might require application of a great deal of domaina building knowledge.
Another objective of segmentation is to perform a change of representation. The pixels of the image must be organized into higher-level units that are either more meaningful or more efficient for further analysis (or both). A critical issue is whether or not segmentation can be performed for many different domains using general bottom-up methods that do not use any special domain knowledge. This chapter presents segmentation methods that have potential use in many different domains. Both region-based and curve-based units are discussed in the following sections. The prospects of having a single segmentation system work well for all problems appear to be dim. Experience has shown that an implementor of machine vision applications must be able to choose from a toolset of methods and perhaps tailor a solution using knowledge of the application.
\section{Motivation}
In recent years, with the advancement of digital era, we are facing a problem of converting handwritten data into digital data for the purpose of saving them in database or repositories in order to analyse or perform operation on the data or keeping log for the future. We still need to process handwritten documents and forms manually that takes lot of times and resources and increases the possibilities to make mistake when processing that data. That causes serious inefficiency and pain worth not to do it manually.

Several offices including government and non-government organizations gets a lot of applications and different kind of forms every day that needs to be processed immediately with efficient measure and more accuracy. 
In this paper we have proposed an efficient way to process those analogue data into digital texts through form segmentation, a technique of processing form image into digital text. Experimental results shows it's accuracy and efficiency to process a form that is lot less time consuming and effective.  

\section{Overview of this book}
In this chapter we have introduced the image segmentation, form segmentation, discuss our objective and motivation. Rest of the chapters are organized as follows. \\
In Chapter \ref{background_study}, we will acquire some background knowledge by discussing about image processing technique, algorithm and tesseract OCR Engine. \\
In Chapter \ref{literature_review}, we are going to discuss about some existing algorithm and technique of Image Processing, tesseract ocr engine working procedure and image processing tool openCV. \\
In chapter \ref{chap:algorithm}, we proposed a working method. In this chapter we discuss about Data Extraction, Data Reorganization and Tesseract Training. \\
In chapter, \ref{chap:result}, we discuss about the performance of our method and output of some sample form.\\
Finally chapter \ref{chap:final}, concludes the document.
